\documentclass[german,headsepline,fleqn,parskip=half]{scrartcl}
\usepackage{fontspec}
\usepackage{microtype}
\usepackage{polyglossia}
\setmainlanguage{german}

\usepackage{scrlayer-scrpage}
\pagestyle{scrheadings}
\ihead{\textbf{Der rote Faden der Vorlesung}\\Mathematische Logik}
\ohead{Robert Schütz\\\today}

\usepackage{enumerate}

\usepackage{amsmath, mathtools, amsthm, amsfonts, amssymb}
\theoremstyle{definition}
\newtheorem*{dfn}{Definition}
\newtheorem*{ex}{Beispiel}
\theoremstyle{plain}
\newtheorem*{thm}{Satz}

\newcommand{\lng}{\mathcal{L}}
\newcommand{\modA}{\mathcal{A}}
\newcommand{\Mod}{\text{Mod}}

\begin{document}
	\begin{dfn}
		Eine \emph{Theorie} $T=(\lng,\Sigma)$ besteht
		\begin{itemize}
			\item aus einer Sprache $\lng=((R_i\vert i\in I);(f_j\vert j\in J);(c_k\vert k\in K))$ und
			\item einer Menge $\Sigma$ von $\lng$-Sätzen.
		\end{itemize}
	\end{dfn}
	
	\begin{dfn}
		Eine $\lng$-Struktur $\modA=(A;(R_i^\modA\vert i\in I);(f_j^\modA\vert j\in J);(c_k^\modA\vert k\in K))$ besteht aus
		\begin{itemize}
			\item einer Menge $A\neq\emptyset$, dem Individuenbereich und
			\item der Interpretation der Relations-, Funktions- und Konstantensymbole. Dabei gilt:
				\begin{align*}
					R_i^\modA&\subseteq A^{n_i} \\
					f_j^\modA&:A^{m_j}\to A \\
					c_k^\modA&\in A
				\end{align*}
		\end{itemize}
	\end{dfn}

	\begin{dfn}
		Eine Struktur $\modA$ ist \emph{Modell} der Theorie $T=(\lng,\Sigma)$,
		falls sie eine $\lng$-Struktur ist und
		$\modA\models\sigma$ für alle $\sigma\in\Sigma$ gilt.
		Das heißt, dass jeder Satz aus $\Sigma$ in dieser Struktur wahr ist ($\modA\models\Sigma$).
		\[\Mod(T):=\{\modA\colon\modA\text{ ist Modell von }T\}\]
		
		Eine Theorie heißt \emph{erfüllbar},
    falls sie ein Modell besitzt ($\Mod(T)\neq\emptyset$).
	\end{dfn}
	
	\begin{ex}
		Die Theorie $T=(\lng(\sim),\{\sigma_1\dots,\sigma_4\})$ der Äquivalenzrelationen mit genau zwei Äquivalenzklassen,
		die durch die Axiome
		\begin{align*}
			\sigma_1&\equiv\forall x(x\sim x) \\
			\sigma_2&\equiv\forall x\forall y(x\sim y\rightarrow y\sim x) \\
			\sigma_3&\equiv\forall x\forall y\forall z(x\sim y\land y\sim z\rightarrow x\sim z) \\
			\sigma_4&\equiv\exists x\exists y\forall z(\lnot(x\sim y)\land(z\sim x\lor z\sim y))
		\end{align*}
		definiert wird, ist erfüllbar,
		denn $\mathcal{Z}=(\mathbb{Z};\{(m,n)\in\mathbb{Z}^2\colon m-n\text{ ist gerade}\})$ ist ein Modell von $T$.
	\end{ex}

	\begin{dfn}[Folgerung]
		$T\models\varphi:\Leftrightarrow\modA\models\varphi\text{ für alle }\modA\in\Mod(T)$.
	\end{dfn}

	\begin{dfn}[Beweis]
		Eine $\lng$-Formel $\varphi$ ist aus $T=(\lng,\Sigma)$ \emph{beweisbar},
		falls es eine \underline{endliche} Folge $\psi_1,\dots,\psi_n$ von $\lng$-Formeln (genannt "Beweis") mit $\psi_n\equiv\varphi$ gibt,
		sodass für jedes $k\in\{1,\dots,n\}$ gilt:
		\begin{itemize}
			\item $\psi_k$ ist ein Axiom oder
			\item $\psi_k\in\Sigma$ oder
			\item $\psi_k$ ist die Konklusion einer Regel
				\[\frac{\chi_1,\dots,\chi_m}{\psi_k}\text{,}\]
				wobei die Prämissen $\chi_1,\dots,\chi_m\in\{\psi_1,\dots,\psi_{k-1}\}$ schon vorher im Beweis auftauchen.
		\end{itemize}
	\end{dfn}

	\begin{dfn}
		content...
	\end{dfn}

	\begin{dfn}
		\begin{align*}
			T=(\lng,\Sigma)\text{ ist konsistent}
			&:\Leftrightarrow\text{Es ex. ein $\lng$-Satz $\sigma$ mit }T\not\vdash\sigma \\
			&\phantom{:}\Leftrightarrow\text{Es ex. kein $\lng$-Satz $\sigma$ mit }T\vdash\sigma\text{ \& }T\vdash\lnot\sigma\text{.}
		\end{align*}
	\end{dfn}

	\begin{thm}[Korrektheitssatz]
		$T\vdash\varphi\Rightarrow T\models\varphi$.
		
		Daraus folgt sofort das Konsistenzlemma: Jede erfüllbare Theorie ist konsistent.
	\end{thm}
	\begin{proof}
		Der Korrektheitssatz folgt aus der Allgemeingültigkeit der Axiome und der Korrektheit der Regeln bzgl. Folgerungen.
		
		Sei $T=(\lng,\Sigma)$ eine erfüllbare Theorie.
		Dann besitzt $T$ ein Modell $\modA$.
		Für jeden $\lng$-Satz $\sigma$ gilt $\modA\models\sigma\Rightarrow\modA\not\models\lnot\sigma$.
		Deshalb gibt es ein $\sigma$ mit $T\not\models\sigma$.
		Angenommen $T\vdash\sigma$ für jeden $\lng$-Satz $\sigma$.
		Dann folgt mit dem Korrektheitssatz $T\models\sigma$ für jeden $\lng$-Satz $\sigma$,
		was ein zu obiger Feststellung Widerspruch ist.
	\end{proof}

	\begin{thm}[Erfüllbarkeitslemma]
		$T$ ist konsistent $\Rightarrow$ $T$ ist erfüllbar.
	\end{thm}

	\begin{thm}[Vollständigkeitssatz]
		$T\models\sigma\Rightarrow T\vdash\sigma$.
		
		Mit dem Korrektheitssatz folgt der Adäquatheitssatz:
		$T\models\sigma\Leftrightarrow T\vdash\sigma$.
	\end{thm}
	\begin{proof}
		\begin{align*}
			T\models\sigma
			&\Rightarrow T\cup\{\lnot\sigma\}\text{ nicht erfüllbar} &\text{(Zshg. zw. Folgerung und Erfüllbarkeit)} \\
			&\Rightarrow T\cup\{\lnot\sigma\}\text{ inkonsistent} &\text{(Kontraposition des Erfüllbarkeitslemmas)} \\
			&\Rightarrow T\vdash\sigma &\text{(Zshg. zw. Beweisbarkeit und Konsistenz)}
		\end{align*}
	\end{proof}

	\begin{thm}[Kompaktheitssatz]
		Eine Theorie $T$ ist genau dann erfüllbar,
		wenn jede \underline{endliche} Teiltheorie $T_0\subseteq T$ erfüllbar ist.
	\end{thm}
	\begin{proof}
		Der ist wichtig; ihr müsst ihn euch aber in den Folien anschauen.
	\end{proof}
	
	\begin{ex}
		Sei $\sigma_n\equiv\underbrace{1+\ldots+1}_{n\text{-mal}}\neq0$.
		Wir wollen zeigen, dass die Theorie
		\[T=(\lng(+,\cdot;0,1),\Sigma\cup\{\sigma_n\colon n\geq1\})\text{,}\]
		der Körper unendlicher Charakteristik,
		wobei $\Sigma$ die üblichen Körperaxiome enthalte,
		erfüllbar ist.
		
		Sei dazu $T_0\subseteq T$ eine endliche Teiltheorie,
		d.h. $T_0=(\lng(+,\cdot;0,1),\Sigma_0)$ mit $\Sigma_0\subseteq\Sigma\cup\{\sigma_n\colon n\geq1\}$ endlich.
		Dann gibt es ein $p\in\mathbb{N}$ mit $\Sigma_0\subseteq\Sigma\cup\{\sigma_n\colon1\leq n<p\}$.
		Sei ohne Einschränkung $p$ eine Primzahl, ansonsten wählen wir einfach die nächsthöhere.
		Weil $\mathbb{Z}/p\mathbb{Z}$ ein Körper mit Charakteristik $p$ ist,
		ist  $\modA=(\mathbb{Z}/p\mathbb{Z};+^\modA,\cdot^\modA;0^\modA,1^\modA)$ ein Modell von $T_0$.
		Also ist $T_0$ erfüllbar und somit ist,
		da $T_0$ eine beliebige endliche Teiltheorie war,
		auch $T$ erfüllbar.
	\end{ex}

	Was bislang noch fehlt: Deduktionstheorem (?), vollständige Theorien, Henkin-Theorie, Termstruktur
	
	Was für die Zukunft noch wichtig wäre: elementar, $\Delta$-elementar,
	elementare Äquivalenz (?), Isomorphie (?)
	
	Außerdem ist es hilfreich, den Beweis des Erfüllbarkeitslemmas (allerdings nicht unbedingt im Detail) zu kennen.
	Der ist allerdings für dieses Format zu lang.
\end{document}
