\documentclass[german,headsepline,fleqn,parskip=half]{scrartcl}
\usepackage{fontspec}
\usepackage{microtype}
\usepackage{polyglossia}
\setmainlanguage{german}

\usepackage{scrlayer-scrpage}
\pagestyle{scrheadings}
\ihead{\textbf{Der rote Faden der Vorlesung}\\Mathematische Logik}
\ohead{Robert Schütz\\\today}

\usepackage{enumerate}

\usepackage{amsmath, mathtools, amsthm, amsfonts, amssymb}
\theoremstyle{definition}
\newtheorem*{dfn}{Definition}
\newtheorem*{ex}{Beispiel}
\theoremstyle{plain}
\newtheorem*{thm}{Satz}
\newtheorem*{lem}{Lemma}

\newcommand{\ag}[1]{\text{ag}[#1]}
\newcommand{\erfb}[1]{\text{erfb}[#1]}
\newcommand{\kd}[1]{\text{kd}[#1]}

\newcommand{\äq}{\text{ äq }}
\newcommand{\impl}{\text{ impl }}

\begin{document}
	\begin{dfn}
		Sei $V$ eine Variablenmenge, d.h. $V\subseteq\{A_0,A_1,\dots\}$.
		Dann heißt eine Abbildung $B:V\to\{0,1\}$ \textbf{Belegung}.
		
		Sei $B:V\to\{0,1\}$ eine Belegung. Dann ist die von $B$ induzierte \textbf{Bewertung} diejenige Abbildung $\hat{B}:F(V)\to\{0,1\}$
		\footnote{Hierbei ist $F(V)=\{\varphi\colon V(\varphi)\subseteq V\}$ die Menge der Formeln, deren Variablen in $V$ liegen.},
		die allen Formeln mit Variablen in $V$ ihren durch die Belegung der Variablen gegebenen Wahrheitswert zuordnet.
		\footnote{Für die induktive Definition der Bewertung einer Formel siehe Kapitel 1, Folie 54.}
		
		Abkürzend schreiben wir oft $B(\varphi)$ statt $\hat{B}(\varphi)$.
	\end{dfn}
	
	\begin{ex}
		Sei $B:\{A,B\}\to\{0,1\}$ die durch $B(A)=1$ und $B(B)=0$ gegebene Bewertung.
		
		Dann ist $B(\varphi)=\hat{B}(\varphi)=0$ für $\varphi\equiv A\rightarrow B$.
	\end{ex}

	\begin{lem}(Koinzidenzlemma (KL))
		Seien $B_1:V_1\to\{0,1\}$, $B_2:V_2\to\{0,1\}$ Belegungen und $\varphi$ eine Formel mit $V(\varphi)\subseteq V$.
		Es gelte $B_1\upharpoonright V(\varphi)=B_2\upharpoonright V(\varphi)$, d.h. $B_1(A)=B_2(A)$ für alle $A\in V(\varphi)$.
		
		Dann ist $B_1(\varphi)=B_2(\varphi)$.
	\end{lem}

	\begin{ex}
		Seien $B_1,B_2:\{A,B,C\}\to\{0,1\}$ die durch $B_1(A)=1,B_1(B)=B_1(C)=0$ und $B_2(A)=B_2(C)=1,B_2(B)=0$ gegebenen Belegungen.
		
		Dann gilt also $B_1(A)=B_2(A)$ und $B_1(B)=B_2(B)$, d.h. $B_1\upharpoonright\{A,B\}=B_2\upharpoonright\{A,B\}$. Weil $V(\varphi)=\{A,B\}$ für $\varphi\equiv A\rightarrow B$ gilt deshalb nach dem Koinzidenzlemma, dass $B_1(\varphi)=B_2(\varphi)$.
	\end{ex}

	\begin{dfn}
		Eine Formel $\varphi$ heißt \textbf{allgemeingültig} ($\ag{\varphi}$) oder eine \textbf{Tautologie}, wenn jede Belegung $B:V(\varphi)\to\{0,1\}$ diese wahr macht, d.h. $B(\varphi)=1$.
		
		Eine Formel $\varphi$ heißt \textbf{erfüllbar} ($\erfb{\varphi}$), wenn es mindestens eine Belegung $B:V(\varphi)\to\{0,1\}$ gibt, die diese wahr macht.
		
		Eine Formel $\varphi$ heißt \textbf{kontradiktorisch} ($\kd{\varphi}$), wenn jede Belegung $B:V(\varphi)\to\{0,1\}$ diese falsch macht.
	\end{dfn}

	\begin{lem}
		\begin{enumerate}[(i)]
			\item $\ag{\varphi}\Rightarrow\erfb{\varphi}$
			\item $\ag{\varphi}\Leftrightarrow\kd{\lnot\varphi}$
			\item $\erfb{\varphi}\Leftrightarrow\text{nicht }\kd{\varphi}$
			\item $\ag{\varphi}\text{ \& }\ag{\psi}\Leftrightarrow\ag{\varphi\land\psi}$
			\item $\ag{\varphi\lor\psi}\Rightarrow\ag{\varphi}\text{ oder }\ag{\psi}$
			\item $\erfb{\varphi}\text{ \& }\erfb{\psi}\Leftrightarrow\erfb{\varphi\land\psi}$
			\item $\erfb{\varphi\lor\psi}\Rightarrow\erfb{\varphi}\text{ oder }\erfb{\psi}$
		\end{enumerate}
	\end{lem}

	\begin{dfn}
		\begin{enumerate}[(i)]
			\item $\varphi\äq\psi:\Leftrightarrow$ für jedes $B:V(\varphi)\cup V(\psi)\to\{0,1\}$ gilt $B(\varphi)=B(\psi)$
			\item $\varphi\impl\psi:\Leftrightarrow$ für jedes $B:V(\varphi)\cup V(\psi)\to\{0,1\}$ mit $B(\varphi)=1$ gilt $B(\psi)=1$
		\end{enumerate}
	\end{dfn}

	\begin{lem}
		\begin{enumerate}[(i)]
			\item $\varphi\äq\psi\Leftrightarrow\ag{\varphi\leftrightarrow\psi}$
			\item $\varphi\impl\psi\Leftrightarrow\ag{\varphi\rightarrow\psi}$
		\end{enumerate}
	\end{lem}

	\begin{lem}(Einsetzungsregel)
		Sei $\varphi$ allgemeingültig und $\psi$ eine beliebige Formel.
		Dann ist auch $\varphi[\psi/A]$ allgemeingültig.
		\footnote{Dabei erhalten wir $\varphi[\psi/A]$, indem wir in $\varphi$ alle Vorkommen von $A$ durch $\psi$ ersetzen.}
	\end{lem}

	\begin{ex}
		Sei $\psi$ eine beliebige Formel.
		Eine Wahrheitstabelle zeigt, dass $\ag{A\lor\lnot A}$ gilt.
		Nach der Einsetzungsregel gilt dann auch $\ag{\psi\lor\lnot\psi}$.
	\end{ex}
\end{document}